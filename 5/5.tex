\documentclass[twocolumn]{ctexart}
\usepackage{amsmath}
\usepackage{tikz}
\usepackage{geometry}
\usepackage{tikzit}
\input{hm.tikzstyles}
\geometry{a4paper, scale=0.8}
\newcommand{\sol}[1]{\subsection*{#1}\noindent\textbf{解:}
	
}
\begin{document}
	\sol{7-1}
	\ctikzfig{fig1}
	
	假设载流导线是均匀的,取导线上顺时针为电流正向,$AB$间较短导线上的电流$I_1$与较长导线上的电流$I_2$之间满足
	$$\frac{I_1}{-I_2}=\frac{2\pi-\theta}{\theta}.$$
	
	取圆弧上一电流元$I\mathrm dl$,设导线圈半径为$R$,由B-S定律,其在$O$点产生的磁感应强度为
	$$\mathrm dB=\frac{\mu_0}{4\pi}\frac{I\mathrm dl}{R^2}=\alpha I\mathrm dl,$$
	
	方向垂直纸面向里为正,其中$\alpha=\mu_0/\left(4\pi R^2\right)$. $O$点磁感应强度
	\begin{align*}
		B_O&=\int_{L}\mathrm dB\\
		&=\int_{0}^{\theta}\alpha I_1R\mathrm d\theta+\int_{\theta}^{2\pi}\alpha I_2R\mathrm d\theta\\
		&=\alpha I_1R\theta+\alpha\frac{\theta}{\theta-2\pi}I_1\left(2\pi-\theta\right)\\
		&=0.
	\end{align*}

	综上,$O$点的磁感应强度为$0$.
	\sol{7-5}
	由带点球面的电势与电荷量关系,有
	$$U-U_{\infty}=\frac{Q}{4\pi\epsilon_0R}.$$
	
	解得
	$$Q=4\pi\epsilon_0 RU.$$
	
	电荷面密度
	$$\sigma=\frac{Q}{4\pi R^2}=\frac{\epsilon_0 U}{R}.$$
	
	按题图,$\theta$处圆环的电荷量为$$\mathrm dq=2\pi\sigma R\sin\theta R\mathrm d\theta.$$
	
	旋转产生的电流
	$$I\mathrm dl=\frac{\mathrm dq}{T}=\frac{\omega\mathrm dq}{2\pi}.$$
	
	由对称性,该电流环在$O$点产生的磁感应强度方向与$\vec\omega$相同. 由B-S定律可得
	\begin{align*}
		\mathrm dB&=2\pi R\sin\theta\times \frac{\mu_0I\mathrm dl}{4\pi R^2}\sin\theta\\
		&=\frac{\mu_0\epsilon_0\omega U}{2}\sin^3\theta\mathrm d\theta.
	\end{align*}
	\begin{align*}
		B&=\int_{0}^{2\pi}\mathrm dB\\
		&=\int_{0}^{\pi}\frac{\mu_0\epsilon_0\omega U}{2}\sin^3\theta\mathrm d\theta\\
		&=\frac{2\mu_o\epsilon_0\omega U}{3}.
	\end{align*}	
	\sol{7-6}
	\ctikzfig{fig2}
	
	设磁通量向球壳表面凸方向为正. 场$\mathbf B$显然是无源的,所以存在矢量场$\mathbf A$,使得$\mathbf B$对任意有边界曲面$S$的通量只取决于$\mathbf A$在边界上的环量$\left(\iint_{S}\mathbf B\cdot\mathrm d\mathbf S=\oint_{\partial S}\mathbf A\cdot\mathrm d\mathbf l\right)$,记半球面为$S_0$,开口圆平面为$S$,由于两个曲面有共同的边界,所以他们的通量相同.
	\begin{align*}
		\Phi&=\iint_{S_0}\mathbf B\cdot\mathrm d\mathbf S_0\\
		&=\iint_{S}\mathbf B\cdot\mathrm d\mathbf S\\
		&=\mathbf B\cdot\mathbf S\\
		&=\left(a\mathbf i+b\mathbf j+c\mathbf k\right)\cdot\left(-\pi R^2\mathbf k\right)\\
		&=-\pi cR^2.
	\end{align*}
	\sol{7-8}
	由对称性得螺绕环内的磁感应强度方向沿圆环方向. 设距螺绕环对称轴$r\left(R_1\leq r\leq R_2\right)$处位置的磁感应强度大小为$B(r)$,由安培环路定理,
	$$2\pi rB(r)=\mu_0NI,$$
	
	解得$B(r)=\mu_0NI/\left(2\pi r\right)$. 铁芯截面的磁通量
	$$\Phi=h\int_{R_1}^{R_2}B(r)\mathrm dr=\frac{\mu_0NIh}{2\pi}\ln\frac{R_2}{R_1}.$$
	\sol{7-9}
	\subsubsection*{(1)}
	\ctikzfig{fig3}
	
	在杆上取线元$\mathrm dl$,到$O$点的距离为$l$,线元因旋转产生的电流为
	$$\mathrm dI=\frac{\lambda\mathrm dl}{T}=\frac{\lambda\omega\mathrm dl}{2\pi}.$$
	
	在$O$点产生的磁感应强度大小
	$$\mathrm dB_O=\frac{\mu_0\mathrm dI}{4\pi l^2}\cdot2\pi l=\frac{\mu_0\lambda\omega\mathrm dl}{4\pi l}.$$
	
	$O$点的磁感应强度大小
	$$B_O=\int_{a}^{a+b}\frac{\mu_0\lambda\omega\mathrm dl}{4\pi l}=\frac{\mu_0\lambda\omega }{4\pi}\ln\frac{a+b}{a}.$$
	
	记垂直纸面向里的单位向量为$\hat{\mathbf{e_z}}$,$O$点的磁感应强度
	$$\mathbf{B_O}=\frac{\mu_0\lambda\omega }{4\pi}\ln\frac{a+b}{a}\hat{\mathbf{e_z}}.$$
	
	\subsubsection*{(2)}
	线元$\mathrm dl$产生的磁矩
	$$\mathrm d\mathbf m=\mathrm dI\cdot\mathbf S=\frac{\lambda\omega\mathrm dl}{2\pi}\cdot\pi l^2\hat{\mathbf{e_z}}=\frac{\lambda\omega\mathrm dl}{2}\cdot l^2\hat{\mathbf{e_z}}.$$
	
	总磁矩
	\begin{align*}
		\mathbf m&=\int_{l=a}^{l=a+b}\mathrm d\mathbf m\\
		&=\int_{a}^{a+b}\frac{\lambda\omega\mathrm dl}{2}\cdot l^2\hat{\mathbf{e_z}}\\
		&=\frac{\lambda\omega\left[\left(a+b\right)^3-a^3\right]}{6}.
	\end{align*}
	
	
	\subsubsection*{(3)}
	当$a\gg b$时,
	\begin{align*}
		\mathbf{B_O}&=\frac{\mu_0\lambda\omega }{4\pi}\ln\frac{a+b}{a}\hat{\mathbf{e_z}}\\
		&\approx\frac{\mu_0\lambda\omega }{4\pi}\cdot\frac{b}{a}\\
		&=\frac{\mu_0\lambda b\omega }{4\pi a}.
	\end{align*}
	\begin{align*}
		\mathbf m&=\frac{\lambda\omega\left[\left(a+b\right)^3-a^3\right]}{6}\\
		&\approx\frac{\lambda\omega a^3}{6}\left[\left(1+\frac{b}{a}\right)^3-1\right]\\
		&=\frac{\lambda\omega a^3}{6}\cdot\frac{3b}{a}\\
		&=\frac{\lambda\omega a^2 b}{2}.
	\end{align*}
	
	\sol{7-10}
	由对称性及B-S定律得,电子所在位置的磁感应强度方向为垂直纸面向里,设其大小为$B$,由安培环路定理,
	$$2\pi\cdot3rB=\mu_0I,$$
	
	解得
	$$B=\frac{\mu_0I}{6\pi r}.$$
	
	$\mathbf v$与磁感应强度方向垂直,由$\mathbf F=-e\mathbf v\times\mathbf B$可得电子所受磁场力的方向垂直导线在纸面内向左,大小
	$$F=\frac{\mu_0Iev}{6\pi r},$$
	
	其中$v=\left|\mathbf v\right|$.
	
	\sol{7-11}
	\ctikzfig{fig4}
	
	由安培环路定理可得距无限长导线$r$处的磁感应强度大小为
	$$B(r)=\frac{\mu_0I_1}{2\pi r},$$
	
	方向垂直导线且可使用相对电流方向的右手螺旋定向.
	
	与无限长导线平行的一段导线所受安培力大小为
	$$F_1=B(d)aI_2=\frac{\mu_0aI_1I_2}{2\pi d}.$$
	
	剩余两段直导线所受的安培力关于$F_1$所在直线对称,设其大小为$F_2$.
	\begin{align*}
		F_2&=\int_{0}^{a}I_2B(d+\frac{\sqrt{3}}{2}l)\mathrm dl\\
		&=\int_{0}^{a}\frac{\mu_0I_1I_2}{2\pi \left(d+\frac{\sqrt{3}}{2}l\right)}\mathrm dl\\
		&=\frac{\sqrt{3}\mu_0I_1I_2}{3\pi}\ln\left(1+\frac{\sqrt{3}a}{2d}\right).
	\end{align*}

	等边三角形导线框所受的安培力总和的大小
	\begin{align*}
		F&=F_1-2F_2\sin30^\circ{}\\
		&=\frac{\mu_0I_1I_2}{\pi}\left[\frac{a}{2d}-\frac{\sqrt{3}}{3}\ln\left(1+\frac{\sqrt{3}a}{2d}\right)\right].
	\end{align*}
	
	方向垂直无限长导线向左.
	
	\sol{7-14}
	\ctikzfig{fig5}
	
	设电流方向从上到下看逆时针为正,圆柱截面半径为$R$,设斜面倾角为$\theta$.如图建立坐标系,则导线框的磁矩
	$$\mathbf m=2lNRI\hat{\mathbf{e_z}}.$$
	
	安培力形成的力偶矩
	$$\mathbf M_1=\mathbf m\times\mathbf B=-2lNRIB\sin\theta\hat{\mathbf{e_y}}.$$
	
	将力偶矩$\mathbf M_1$平移至点$A$,重力在点$A$产生的力矩
	$$\mathbf M_g=mgR\sin\theta\hat{\mathbf{e_y}}.$$
	
	由力矩平衡$\mathbf M_1+\mathbf M_g=0$得
	$$I=\frac{mg}{2NlB}.$$
	
	\sol{7-15}
	\ctikzfig{fig6}
	
	由对称性易得磁感应强度的方向如图所示.如图按角度分割,设其产生的磁感应强度在合场强方向的投影为$\mathrm dB_y$,
	$$\mathrm dB_y=\frac{\mu_0}{2\pi R}\cdot\frac{I\mathrm d\theta}{\pi}\cdot\sin\theta.$$
	
	半圆同轴线上磁感应强度大小
	\begin{align*}
		B&=\int_{0}^{\pi}\mathrm dB_y\\
		&=\int_{0}^{\pi}\frac{\mu_0}{2\pi R}\cdot\frac{I\mathrm d\theta}{\pi}\cdot\sin\theta\\
		&=\frac{\mu_0I}{\pi^2R}.
	\end{align*}

	方向如图所示.
	
	\sol{7-17}
	\ctikzfig{fig7}
	
	设导线框的质心到$OO'$轴的距离为$r_c$,正方形边长为$a$,线框的正方形三边的总质量为$3m$. 两个径向段的质心位置$r_{c1}=a/2$,质量为$m_1=2m$,与$OO'$平行段杆的质心位置为$r_{c2}=a$,质量为$m_2=m$,总的质心位置为$r_c=\left(r_{c1}m_1+r_{c2}m_2\right)/(3m)=2a/3$.
	
	导线框所受安培力如图,大小为$BIa$.由力矩平衡
	$$BIa\cdot a\cos\alpha=3mgr_c\sin\alpha$$
	
	得
	$$B=\frac{3mgr_c\tan\alpha}{a^2I}=\frac{2mg\tan\alpha}{aI},$$
	
	代入$m=\rho Sa$得
	$$B=\frac{2\rho Sg\tan\alpha}{I},$$
	
	代入数值$S=2.0\ \mathrm{mm^2}$,$\rho=8.9\ \mathrm{g/cm^3}$,$I=10\ \mathrm A$,$\alpha=15^\circ{}$,$g=9.8\ \mathrm{m/s^2}$得
	$$B=9.35\ \mathrm{mT}.$$
\end{document}