\documentclass[twocolumn]{ctexart}
\usepackage{amsmath}
\usepackage{mathrsfs}
\usepackage{tikz}
\usepackage{geometry}
\usepackage{tikzit}
\input{hm.tikzstyles}
\geometry{a4paper, scale=0.8}
\newcommand{\sol}[1]{\subsection*{#1}\noindent\textbf{解:}
	
}
\newcommand{\un}[1]{\ \mathrm{#1}}
\begin{document}
\sol{11-2}
已知光程差为$\delta=\lambda/3$,相位差
$$\Delta\varphi_{12}=\frac{\delta}{\lambda}\cdot2\pi=\frac{2\pi}{3}.$$

设光在光屏的强度为$I$,则$P$点的光强
$$I_P=I+I+2I\cos\Delta\varphi_{12}=I.$$

干涉相长达到最大光强时
$$I_{max}=I+I+2I=4I.$$

故$I/I_{max}=1/4$.

\sol{11-3}
设液体的折射率为$n$.

在空气中,明条纹之间的距离
$$\Delta x=\frac{D}{d}\lambda.$$

液体中波长变为$\lambda/n$,
$$\Delta x'=\frac{D}{nd}\lambda.$$

由题意,$3\Delta x=4\Delta x'$,解得
$$n=\frac{4}{3}.$$

\sol{11-7}
记厚度$h=1.2\times10^{-7}\un m$,折射率$n=1.33$. 因干涉而加强的波长$\lambda$满足
$$2nh+\frac{1}{2}\lambda=k\lambda,k=1,2,\cdots.$$

整理出$\lambda$,代入数值,得
$$\lambda=\frac{319.2\un{nm}}{k-\frac{1}{2}},k=1,2,\cdots.$$

当且仅当$k=1$时,$\lambda$在可见光范围
$$\lambda=638.4\un{nm}.$$

\sol{11-11}
设明纹之间的距离为$\Delta x$,之间空气劈尖厚度的变化$\Delta h=\theta\Delta x$,两明纹之间光程差为一个波长
$$2n\Delta h=\lambda.$$

整理得$\Delta x=\lambda/(2n\theta)$.已知$\lambda=500.0\un{nm}$.对于空气劈尖,$n_0=1$,$\Delta x_0=\frac{250}{\theta}\un{nm}$.对于充满液体的劈尖,$n=1.40$,$\Delta x=\frac{1250}{7\theta}\un{nm}$.劈棱处为暗纹,由题意,有$\left(4+\frac{1}{2}\right)\left(\Delta x_0-\Delta x\right)=1.61\un{mm}$,解得
$$\theta=2.00\times10^{-4}\un{rad}.$$

\sol{11-13}

光程差的变化$\Delta\delta=7\lambda$,设薄膜厚度为$h$,又有$\Delta\delta=2nh-2h$,解得
$$h=\frac{7\lambda}{2(n-1)},$$

代入数值得$h=5.03\un{\mu m}$.

\sol{11-16}
\subsection*{(1)}
半径为$r$处空气薄膜的厚度
$$h=R-\sqrt{R^2-r^2}\approx\frac{r^2}{2R}.$$

牛顿环中央为暗纹,第$k$个亮环处空气劈尖内产生的光程差为$\left(k-\frac{1}{2}\right)\lambda$,设第$k$级暗环的半径为$r_k$,由
$$\left(k-\frac{1}{2}\right)\lambda=2\frac{r_k^2}{2R}$$

得$r_k=\sqrt{\left(k-\frac{1}{2}\right)R\lambda}$.

\subsection*{(2)}
折射率为$n_2=1.5$的一半在一个介面上产生半波损失,折射率为$n_3=1.75$的另一半上有两个介面产生半波损失,牛顿环图样由同心交错的明暗圆环变为接近下图所示:
\ctikzfig{fig1}

\sol{11-17}
使用菲涅尔半波带法,暗纹的衍射角$\varphi$满足
$$a\sin\varphi=\pm2k\frac{\lambda}{2},k=1,2,3,\cdots.$$

其中$k$是暗纹的级数.$+3$级暗条纹的衍射角$\varphi_3$满足
$$\sin\varphi_3=\frac{3\lambda}{a}.$$

代入数值得$\sin\varphi_3=0.01$.正负三级暗纹之间的距离为
$$d=2f\tan\varphi_3\approx2f\sin\varphi_3,$$

代入数值得$d=0.008\un m$.

\sol{11-19}
对于衍射角为$\varphi$,且$\varphi$很小的光,其最大光程差
$$\delta_{max}=a\sin\varphi.$$

中央明纹的角分布的边界对应的$\delta_{max}$为$\pm\lambda$,即$\sin\varphi_0=\pm\frac{\lambda}{a}$,是第一级暗纹的位置.中央明纹的角宽度$\Delta\varphi_0=2\arcsin\frac{\lambda}{a}\approx\frac{2\lambda}{a}$,代入数值得
$$\Delta\varphi_0=5.46\times10^{-3}\un{rad}=0.313^\circ{}.$$

第$k$级暗纹的角位置$\varphi_{kd}$满足
$$a\sin\varphi_{kd}=\pm k\lambda.$$

第$k$级明纹的角宽度
\begin{align*}
\Delta\varphi_k&=\varphi_{(k+1)d}-\varphi_{kd}\\
&=\arcsin\frac{(k+1)\lambda}{a}-\arcsin\frac{k\lambda}{a}\\
&\approx\frac{\lambda}{a},k=1,2,3,\cdots.
\end{align*}

代入数值得$\Delta\varphi_k=2.73\times10^{-3}\un{rad}=0.156^\circ{}$.

中央明纹的线宽度
$$\Delta x_0=f\Delta\varphi_0,$$

代入数值得$\Delta x_0=2.73\un{mm}$.同理得第$k$级明纹的线宽度$\Delta x_k=1.37\un{mm}$.

\sol{11-20}
光栅常数
$$d=\frac{1\un{cm}}{1000}=10^{-5}\un m.$$

第二级明条纹的衍射角$\varphi$满足
$$d\sin\varphi=\pm2\lambda.$$

解得$\sin\varphi=\pm2\lambda/d=0.1$.记光栅到屏的距离为$D$,第二级明条纹到中央明条纹的距离为$x$,则有几何关系
$$x=D\tan\varphi=D\frac{\sin\varphi}{\sqrt{1-\sin^2\varphi}}.$$

代入数值得$x=0.101\un m$.

\sol{11-22}
两种光的明条纹的衍射角$\varphi_1$和$\varphi_2$满足
$$d\sin\varphi_1=\pm k\lambda_1,k=0,1,2,\cdots.$$
$$d\sin\varphi_2=\pm k\lambda_2,k=0,1,2,\cdots.$$

设两种波长的谱线除中央明纹外第二次重合的级数分别为$k_1$和$k_2$,那么有
$$k_1\lambda_1=k_2\lambda_2.$$

代入数值,得$2k_1=3k_2$.由于两种波长的谱线是除中央明纹外第二次重合,应有$k_1=6$,$k_2=4$.光栅常数

$$d=\frac{k_1\lambda_1}{\sin60^\circ{}}.$$

代入数值得$d=3.048\un{\mu m}$.

\sol{11-24}
先考虑多缝干涉,光栅常数$d=(1/500)\un{mm}=2\times10^{-6}\un m$.相邻两缝衍射角为$\theta$的光线的光程差
$$\delta=d\left(\sin\varphi+\sin\theta\right).$$

其中$\varphi=30^\circ{}$.如题图所示,$\sin\theta$的取值范围是$(-1,1)$,对应$\delta$的取值范围是$\left(-\frac{1}{2}d,\frac{3}{2}d\right)$.由光程差和亮条纹级数的关系
$$\delta=k\lambda,k=0,\pm1,\pm2,\cdots$$

得$k$的取值范围是$\left(-\frac{d}{2\lambda},\frac{3d}{2\lambda}\right)$.同时有
\begin{align}\sin\theta=\frac{k\lambda}{d}-\frac{1}{2},k=0,\pm1,\pm2,\cdots.\end{align}
代入数值得$k$的范围是$\left(-1.7,5.1\right)$.

设单条狭缝衍射产生的暗纹的衍射角为$\theta_d$,则有$a\left(\sin\varphi+\sin\theta_d\right)=k\lambda,k=0,\pm1,\pm2,\cdots$,即
\begin{align}\sin\theta_d=k\frac{\lambda}{a}-\frac{1}{2},k=0,\pm1,\pm2,\cdots.\end{align}

令$\theta=\theta_d$,由(1)和(2)得
$$k=k_d\frac{d}{a}=2k_d.$$

其中$k$和$k_d$分别为多缝干涉的亮纹级数和单缝衍射的暗纹级数.当$k=2k_d,k_d=0,\pm1,\pm2,\cdots$时,发生缺级.

综上,最多能观察到$+5$级谱线,能观察到的谱线有$-1,0,+1,+3,+5$级谱线.

\sol{11-26}

设自然光的强度为$I$,线偏光的强度为$kI$.自然光经过偏振片后光强变为$I/2$.由题意有
$$\frac{I}{2}+kI=5\frac{I}{2},$$

解得$k=2$.线偏光强度占总强度的比例
$$\eta_k=\frac{k}{1+k}=\frac{2}{3}.$$

自然光占总强度的比例
$$\eta=\frac{1}{3}.$$

\sol{11-27}
设从第一块偏振片透射的偏振光的强度为$I_0$,则$I_1=I_0\cos^2 30^\circ{}=\frac{3}{4}I_0$.偏振化方向夹角变为$45^\circ{}$后,透射光强度$I_2=I_0\cos^2 45^\circ=\frac{1}{2}I_0$.比较得$I_2=\frac{2}{3}I_1$.

\sol{11-29}
光从水中射向玻璃而反射时,起偏角
$$i_1=\arctan\frac{1.50}{1.33}=48.4^\circ{}.$$

光从玻璃中射向水而反射时,起偏角
$$i_2=\arctan\frac{1.33}{1.50}=41.6^\circ{}.$$
\end{document}