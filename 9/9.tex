\documentclass[twocolumn]{ctexart}
\usepackage{amsmath}
\usepackage{mathrsfs}
\usepackage{tikz}
\usepackage{geometry}
\usepackage{tikzit}
\input{hm.tikzstyles}
\geometry{a4paper, scale=0.8}
\newcommand{\sol}[1]{\subsection*{#1}\noindent\textbf{解:}
	
}
\newcommand{\un}[1]{\ \mathrm{#1}}
\begin{document}

\section*{第12章}
\sol{12-1}
由光速不变原理得固有长度
$$L_0=c\Delta t.$$

\sol{12-2}
记某参考系相对飞船系的运动速度为$-u_{r1}$,微流星在某参考系中的速度为$-u_{r2}$. 微流星在飞船系中的速度大小
$$u=\frac{-u_{r1}-u_{r2}}{1+\frac{u_{r1}u_{r2}}{c^2}}=0.981c.$$

经过飞船所耗时间
$$t=\frac{l}{u}=1.19\un{\mu s}.$$

\sol{12-5}
设$S'$相对$S$速度为$v_{r}$,由
$$\gamma=\frac{1}{\sqrt{1-\frac{v^2}{c^2}}}=\frac{2000\un m}{1000\un m}=2,$$

解得$v_r=\frac{\sqrt{3}}{2}c$.

时间间隔
\begin{align*}
\Delta t_{21}&=\frac{t_2-\frac{v_r}{c^2}x_2}{\sqrt{1-\frac{v_r^2}{c^2}}}-\frac{t_1-\frac{v_r}{c^2}x_1}{\sqrt{1-\frac{v_r^2}{c^2}}}\\
&=\frac{-\Delta x_{21}\frac{v_r}{c^2}}{\sqrt{1-\frac{v_r^2}{c^2}}}
\end{align*}

代入数值得
$$\Delta t_{21}=-5.77\un{\mu s}.$$

即从$S'$相对$S$的运动方向来看位置靠前的事件比靠后的的事件早发生$5.77\un{\mu s}$.

\sol{12-7}
记两系的$A$和$A'$钟位置为各自的原点,两者重合时$t=t'=0$. 设$\Delta x$为从$A$到$B$的有向距离. $A'$钟和$B$钟相遇时$S$系的时间即为$B$钟读数
$$t_B=\frac{\Delta x}{v}.$$

$S'$系中$A'$钟读数
\begin{align*}
t_{A'}&=\frac{t_B-\frac{v}{c^2}\Delta x}{\sqrt{1-\frac{v^2}{c^2}}}\\
&=\frac{\Delta x}{v}\sqrt{1-\frac{v^2}{c^2}}.
\end{align*}

\sol{12-8}
在洛伦兹变换
$$
\left\{
\begin{matrix}
	x=\gamma(x'+\beta ct')\\
	ct=\gamma(ct'+\beta x')
\end{matrix}
\right.
$$
中固定时间$t'$,整理得
$$ct=\beta x+\frac{c}{\gamma}t'$$

在以$x$为横轴,$ct$为纵轴的坐标系中,在$S'$系中同时发生的两个事件所在直线的斜率为$\beta$.
$$\beta=c\frac{t_2-t_1}{x_2-x_1}.$$

又从以上洛伦兹变换方程组的第一式得,当$\Delta t'=0$时,
$$\Delta x=\gamma\Delta x'=\frac{1}{\sqrt{1-\beta^2}}\Delta x'.$$

\begin{align*}
	\Delta x'&=\sqrt{1-\beta^2}\left(x_2-x_1\right)\\
	&=\sqrt{1-c^2\frac{(t_2-t_1)^2}{(x_2-x_1)^2}}(x_2-x_1)\\
	&=\left(\Delta x^2-c^2\Delta t^2\right)^\frac{1}{2}.
\end{align*}

\sol{12-9}
火箭发射时,地面上的时间
$$\Delta t_1=\gamma\Delta t'=12.5\un s.$$

火箭升空的距离
$$x=v\Delta t_1=12.5\un s\cdot0.6c.$$

导弹落地所用时间
$$\Delta t_2=\frac{x}{v_1}=25\un s.$$

从火箭发射到导弹到达地球的时间
$$\Delta t=\Delta t_1+\Delta t_2 = 37.5\un s.$$

\sol{12-10}
记$\mu$子动能为$E_k=10395\un{MeV}$. 总能量
\begin{align*}
	E&=\gamma m_0c^2\\
	&=m_0c^2+(\gamma-1)m_0c^2\\
	&=m_0c^2+E_k\\
	&=10500\un{MeV}.
\end{align*}

由$E_k=(\gamma-1)m_0c^2$得$\gamma=100$. 解得
$$v=\frac{3\sqrt{1111}}{100}c=0.99995c.$$

动量
$$p=\gamma m_0v=1.05\times 10^{4}\un{MeV\cdot c^{-1}}.$$

\sol{12-11}
由动量守恒,粒子正碰合并后静止,所有能量均转化为静质能. 总能量
$$E=2\frac{m_0c^2}{\sqrt{1-\frac{v^2}{c^2}}}.$$

由$E=m_0'c^2$得
$$m_0'=\frac{2m_0}{\sqrt{1-\frac{v^2}{c^2}}}.$$

\sol{12-13}
\subsection*{(1)}
电子的总能量
$$E=\frac{m_ec^2}{\sqrt{1-\frac{v^2}{c^2}}}=5.81\times10^{-13}\un J.$$
\subsection*{(2)}
电子的经典力学动能与相对论动能之比
$$\frac{E_{k0}}{E_k}=\frac{\frac{1}{2}m_0v^2}{(\gamma-1)m_0c^2}=\frac{v^2}{2(\gamma-1)c^2}=0.08.$$

\sol{12-16}
记$E_{kB}=6m_0c^2$,复合质点为$C$. 由相对论能量与动量的关系
$$\left(E_{kB}+m_0c^2\right)^2=p_B^2c^2+m_0^2c^4,$$

动量守恒
$$p_B=p_C,$$

其中
$$p_C=\frac{m_Cv_C}{\sqrt{1-\frac{v_C^2}{c^2}}},$$

能量守恒
$$E_{kB}+m_0c^2+m_0c^2=\frac{m_Cc^2}{\sqrt{1-\frac{v_C^2}{c^2}}}$$

解得$m_C=4m_0$.

\section*{第13章}
\sol{13-2}
由
$$\lambda_1T_1=b,$$
$$\lambda_2T_2=b,$$
$$E_1=\sigma T_1^4,$$
$$E_2=\sigma T_2^4$$

得总辐射出射度变为原来的$\frac{E_2}{E_1}=\frac{\lambda_1^4}{\lambda_2^4}=3.63$倍.

\sol{13-3}
\subsection*{(1)}
设金属的逸出功为$A$,由光电效应方程
$$h\nu=\frac{1}{2}mv_m^2+A,$$

遏止电压$U_e$满足$eU_e=\frac{1}{2}mv_m^2$得
\begin{equation}
	U_e=\frac{h}{e}\nu-\frac{A}{e}.
\end{equation}


故$AB$线斜率$k=\frac{h}{e}$,与金属材料无关.
\subsection*{(2)}
由图得斜率$K=\frac{2\un V}{5.0\times10^{14}\un {Hz}}=4\times10^{-15}\un{V/Hz}$. 由(1)式得$h=eK=6.4\times10^{-34}\un{J\cdot s}$.

\sol{13-5}
正确的有(2)和(4).

\sol{13-8}
\subsection*{(1)}
散射前光的波长
$$\lambda_0=\frac{hc}{\varepsilon}=124.3\un{pm}.$$

散射过程中波长的变化量
$$\Delta\lambda=\frac{h}{m_ec}(1-\cos60^\circ)=1.2\un{pm}.$$

碰撞后散射光的波长$\lambda=\lambda_0+\Delta\lambda=125.5\un{pm}$.

\subsection*{(2)}
由能量守恒
$$\frac{hc}{\lambda_0}=\frac{hc}{\lambda}+E_{ke}$$

解得$E_{ke}=1.53\times10^{-17}\un J$.

\sol{13-9}
记$\varepsilon=0.5\un{MeV}$,$E_e=0.1\un{MeV}$. 由
\begin{align*}
	&\lambda_0=\frac{hc}{\varepsilon}\\
	&hc\left(\frac{1}{\lambda_0}-\frac{1}{\lambda}\right)=E_e\\
	&\lambda=\lambda_0+\Delta\lambda
\end{align*}

由上式整理得
$$\frac{\Delta\lambda}{\lambda}=\frac{\Delta\lambda}{\Delta\lambda+\lambda_0}=\frac{E_e}{\varepsilon}=\frac{1}{5}.$$

故
$$\frac{\Delta\lambda}{\lambda_0}=\frac{1}{4}.$$

\sol{13-10}
记$\lambda_0=3\un{pm}$,$E_{ke}=(\gamma-1)m_ec^2=\frac{1}{4}m_ec^2$.
由能量守恒
$$\frac{hc}{\lambda_0}=\frac{hc}{\lambda}+E_{ke}.$$

解得$\lambda=4.34\un{pm}$,则$\Delta\lambda=1.34\un{pm}$.

设散射角为$\varphi$,由
$$\Delta\lambda=\frac{h}{m_ec}(1-\cos\varphi)$$

得$\varphi=63^\circ{}$.

\sol{13-16}
电子的动量
$$p_e=h/\lambda_e=3.32\times10^{-24}\un{kg\cdot m\cdot s^{-1}},$$

总能量$E_e=\sqrt{p_e^2c^2+m_e^2c^4}=8.20\times10^{-14}\un J$. 动能
$$E_{ke}=E_e-m_ec^2=6.03\times10^{-18}\un J.$$

光子的动量
$$p_p=h/\lambda_p=3.32\times10^{-24}\un{kg\cdot m\cdot s^{-1}},$$

动能
$$E_{kp}=p_pc=9.95\times10^{-16}\un J.$$

\sol{13-18}
粒子的动量
$$p=\frac{h}{\lambda}=3.32\times10^{-22}\un{kg\cdot m\cdot s^{-1}}.$$

记$U=206\un V$. 由
$$eU=\frac{1}{2}mv^2$$

和
$$\frac{1}{2}mv^2=\frac{p^2}{2m}$$

解得$m=\frac{h^2}{2eU\lambda^2}$.

代入数值得粒子的静质量$m_0=1.67\times10^{-33}\un{kg}$.

\sol{13-21}
\subsection*{(1)}
动量的不确定度
$$\Delta p_e=m_e\Delta v=9.1\times10^{-33}\un{kg\cdot m\cdot s^{-1}}.$$

坐标的不确定度
$$\Delta x_e=\frac{h}{\Delta p_e}=0.072\un m.$$

\subsection*{(2)}
同(1)得
$$\Delta x_2=\frac{h}{m_3\Delta v}=6.63\times10^{-19}\un m.$$

\subsection*{(3)}
同(1)得
$$\Delta x_3=\frac{h}{m_3\Delta v}=6.63\times10^{-28}\un m.$$

\sol{13-23}
$$\left|\Psi(x)\right|^2=\frac{2}{a}\sin^2\left(\frac{2\pi}{a}x\right).$$

联系$a$的范围可以知道,当$x$取$a/4$及$3a/4$时,$\left|\Psi(x)\right|^2$最大,粒子被发现的概率最大.

\sol{13-24}
$$\left|\Psi(x)\right|^2=c^2x^2(l-x)^2,0\leq x\leq l,$$

归一化
$$\int_{-\infty}^{+\infty}\left|\Psi(x)\right|^2\mathrm dx=\int_0^l\left|\Psi(x)\right|^2\mathrm dx=1,$$

解得
$$c^2=\frac{30}{l^5}.$$

在$[0,l/3]$发现粒子的概率
$$\int_0^{\frac{l}{3}}\left|\Psi(x)\right|^2\mathrm dx=\frac{17}{81}=0.21.$$

\end{document}